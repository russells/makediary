\documentclass{article}
% Copyright 2002-2004 Donald Bindner
% Copyright      2004 Russell Steicke
% Permission is granted to copy, distribute and/or modify this
% document under the terms of the GNU Free Documentation License,
% Version 1.1 or any later version published by the Free Software
% Foundation.

\usepackage{multicol}

\setlength{\textheight}{11 in}
\setlength{\textwidth}{7.5 in}
\setlength{\hoffset}{-2 in}
\setlength{\voffset}{-1 in}
\setlength{\footskip}{12 pt}
\setlength{\oddsidemargin}{1.5 in}
\setlength{\evensidemargin}{1.5 in}
\setlength{\topmargin}{13 pt}
\setlength{\headheight}{12 pt}
\setlength{\headsep}{0 in}

\setlength{\parindent}{0 in}

\ifx \pdfpagewidth \undefined
\else
 \pdfpagewidth=8.5in   % page width of PDF output
 \pdfpageheight=11in   % page height of PDF output
\fi

\newcommand{\ctrl}{C-}

\begin{document}
\thispagestyle{empty}
\fontsize{9}{10}\selectfont

\newcommand{\cmd}[2]{#1 \hfill \texttt{#2}\par}
\newcommand{\head}[1]{{\large\textbf{#1}}\\}

\begin{multicols}{2}
%{\Large Unix commands}

%\vskip 15pt

\vbox{\head{Getting help}
Yes, there is help available.\par
\vskip 5pt
\cmd{Search the man pages for a keyword}{man -k {\it word}}
\cmd{Read the man page for a subject}{man {\it subject}}
\cmd{Read a man page in a particular section}{man {\it n} {\it subject}}
\cmd{(Man sections: 1=User~commands 2=System~calls 3=Functions 5=File~formats
8=Admin~commands)}{}
\cmd{Info system (mostly for programs from the GNU project)\par}%
{info {\it subject}}
\cmd{Many programs take a {\tt -h} or {\tt --help} option.}{}
}

\vskip 10pt

\vbox{\head{Directory navigation}
Moving around the file system.\par
\vskip 5pt
\cmd{Where am I?}{pwd}
\cmd{Change directory}{cd {\it dir}}
\cmd{Go to home directory}{cd}
\cmd{Up one level}{cd ..}
\cmd{Return to prev dir}{cd -}
}

\vskip 10pt

\vbox{\head{Directory listing}
Listing files in your file system.\par
\vskip 5pt
\cmd{Basic file listing}{ls}
\cmd{File listing including hidden (dot) files}{ls -a}
\cmd{Long listing (includes time, owner, etc)}{ls -l}
\cmd{List including char for file type}{ls -F}
\cmd{View directory, not its contents}{ls -d}
}

\vskip 10pt

\vbox{\head{Finding files}
Finding files and operating on the files you do find.\par
\vskip 5pt
\cmd{Find files named {\it pattern}}{find {\it dir} -name {\it pattern}}
\cmd{Find files, run command on them\par}%
{find {\it dir} -name {\it pattern} | xargs {\it command}}
}

\vskip 10pt

\vbox{\head{Date and time}
Find and set the current date and time, and get run time of commands.\par
\vskip 5pt
\cmd{Show the current date and time}{date}
\cmd{Show the current date and time in UTC}{date -u}
\cmd{Show date and time in RFC822 format}{date -R}
\cmd{Show date and time in specified format}{date +{\it spec}}
\cmd{}{date +\%Y\%m\%d {$=>$} 20041004}
\cmd{Set the system date and time (root only)}{date -s {\it datespec}}
\cmd{Show run time summary when command completes\par}{time {\it command}}
}

\vskip 10pt

\vbox{\head{Managing free space}
Commands to find out how much space is used or free.\par
\vskip 5pt
\cmd{List space used on mounted file systems}{df}
\cmd{List space used on one file system}{df \it filesystem}
\cmd{List space used in human-readable format}{df -h}
\cmd{List space used in a directory}{du \it dir}
\cmd{Summary of space used in a directory}{du -s \it dir}
\cmd{Summary in human-readable format}{du -hs \it dir}
}

\vskip 10pt

\vbox{\head{File system management}
Commands to mount, unmount and check file systems.\par
\vskip 5pt
\cmd{Mount a file system listed in fstab}{mount \it filesystem}
\cmd{}{mount \it device}
\cmd{Mount a file system}{mount -t {\it fstype} {\it device} {\it dir}}
\cmd{Unmount a mounted file system}{umount \it filesystem}
\cmd{Check an unmounted file system}{fsck -t {\it type} {\it device}}
\cmd{Create a file system on a device}{mkfs -t {\it type} {\it device}}
\cmd{List space used in human-readable format}{df -h}
\cmd{List space used in a directory}{du \it dir}
\cmd{Summary of space used in a directory}{du -s \it dir}
}

\vskip 10pt

\vbox{\head{Managing users}
Commands to manage users and passwords.\par
\vskip 5pt
\cmd{Change your own password}{passwd}
\cmd{Change password for a user (root only)}{passwd \it username}
\cmd{Change user id}{su \it username}
\cmd{Change user id, run login shell}{su - \it username}
\cmd{Change user id to root}{su}
\cmd{Run single command as root}{sudo \it command}
\cmd{Run single command, with shell redirection (your shell is running as you,
so it may not be able to open some files to do redirections)\par}%
{sudo sh -c '{\it command} > {\it f}'}
}

\vskip 10pt

\vbox{\head{Editors}
Text editors.\par
\vskip 5pt
\cmd{Visual editor, available everywhere}{vi [{\it filename}]}
\cmd{Vi IMproved, on most GNU/Linux systems}{vim}
\cmd{Emacs, has a kitchen sink and also a text editor}{emacs}
\cmd{Simpler text editor, from pine}{pico}
\cmd{Pico clone}{nano}
}

\vskip 10pt

\vbox{\head{Archiving}
Commands to archive files and directories.\par
\vskip 5pt
\cmd{Create a tar archive}{tar cf {\it archivefile.tar} {\it dirs}}
\cmd{Extract contents of a tar archive}{tar xf {\it archivefile.tar}}
\cmd{Create a compressed tar archive}{tar czf {\it archivefile.tar.gz} {\it dirs}}
\cmd{Copy a directory tree\par}%
{( cd {\it src} \&\& tar cf - . ) | ( cd {\it dest} \&\& tar xf -)}
}

\vskip 10pt

\vbox{\head{Mail}
Commands to send and read mail.  The Unix facilities for reading mail are
surprisingly rich, and often much more convenient than graphical mail
readers.\par
\vskip 5pt
\cmd{Send mail from command line}{mail -s {\it subject} {\it recipients}}
\cmd{}{{\it commands} | mail -s {\it subject} {\it recipients}}
\cmd{Popular text mode mail reader}{mutt}
\cmd{Send mail with mutt from command line\par}%
{echo hello | mutt -x -s {\it subject} [-a {\it file}] {\it recipients}}
\cmd{Older (non-free) text mode mail reader}{pine}
}

\vskip 10pt

\end{multicols}

\vspace{\fill}
\copyright 2002-2004 Donald J.\ Bindner,
\copyright 2004 Russell Steicke
-- licensed under the terms of the GNU
Free Documentation License version 1.1 or later.
\end{document}


