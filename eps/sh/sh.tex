\documentclass{article}
% Copyright 2002-2004 Donald Bindner
% Copyright      2004 Russell Steicke
% Permission is granted to copy, distribute and/or modify this
% document under the terms of the GNU Free Documentation License,
% Version 1.1 or any later version published by the Free Software
% Foundation.

\usepackage{multicol}

\setlength{\textheight}{11 in}
\setlength{\textwidth}{7.5 in}
\setlength{\hoffset}{-2 in}
\setlength{\voffset}{-1 in}
\setlength{\footskip}{12 pt}
\setlength{\oddsidemargin}{1.5 in}
\setlength{\evensidemargin}{1.5 in}
\setlength{\topmargin}{13 pt}
\setlength{\headheight}{12 pt}
\setlength{\headsep}{0 in}

\setlength{\parindent}{0 in}

\ifx \pdfpagewidth \undefined
\else
 \pdfpagewidth=8.5in   % page width of PDF output
 \pdfpageheight=11in   % page height of PDF output
\fi

\newcommand{\ctrl}{C-}

\begin{document}
\thispagestyle{empty}
\fontsize{9}{10}\selectfont

\newcommand{\cmd}[2]{#1 \hfill \texttt{#2}\par}
\newcommand{\head}[1]{{\large\textbf{#1}}\\}

\begin{multicols}{2}
%{\Large Unix commands}

%\vskip 15pt

\vbox{\head{Environment variables}
Setting and using environment variables.  The shell keeps these internally,
unless you export them, in which case they are then also available to child
processes.\par
\vskip 5pt
\cmd{Setting a variable.  {\it value} must be quoted if it contains white space
or shell metachars.}{VAR={\it value}}
\cmd{Getting a variable's value}{\$VAR {\rm or} "\$VAR"}
\cmd{Export a varialbe to sub-processes}{export VAR}
\cmd{Expand shell script arguments separately}{"\$@"}
\cmd{Substitute default value if {\tt VAR} is unset}{\$\{VAR:-{\it default}\}}
\cmd{Assign substitute value if {\tt VAR} is unset}{\$\{VAR:={\it default}\}}
\cmd{Assign shell variable array}{A=({\it element1} {\it element2} ...)}
\cmd{Subscript shell array}{\$\{A[{\it n}]\}}
\cmd{Expand entire array}{\$\{A[*]\}}
\cmd{Expand array, quoting each element separately}{"\$\{A[@]\}"}
}

\vskip 10pt

\vbox{\head{Shell settings}
Ways to change the behaviour of the shell.
\vskip 5pt
\cmd{Make shell exit if any command exits non-0}{set -e}
\cmd{Echo every command before executing it}{set -x}
\cmd{Enable vi-like or emacs-like command line editing\par}%
{set -o vi{\rm , } set -o emacs}
\cmd{Set command line args (useful in shell scripts in combination with
{\it getopt(1)} or the getopts builtin)}{set -- {\it args}}
}

\vskip 10pt

\vbox{\head{Shell hints}
Hints for using the shell.  Here are some of the constructs that let you create
powerful shell programs, and they can all be used on the command line.\par
\vskip 5pt
\cmd{Run command in background}{{\it command} \&}
\cmd{Redirect output}{{\it command} > {\it file}}
\cmd{Redirect input}{{\it command} < {\it file}}
\cmd{Redirect output and error output}{{\it command} >{\it file} 2>\&1}
\cmd{Run command2 only if command1 succeeds (0 exit status)\par}%
{{\it command1} \&\& {\it command2}}
\cmd{Run command2 only if command1 fails (non-0 exit status)\par}%
{{\it command1} || {\it command2}}
\cmd{Force a command to succeed}{{\it command} || :}
\cmd{Run commands in subshell}{( {\it commands} )}
\cmd{Pipe between two subshells}{( {\it commands} ) | ( {\it commands} )}
\cmd{Substitute command output}{` {\it command} ` {\rm or} \$( {\it command} )}
\cmd{Shell arithmetic}{\$((3*4))}
\cmd{}{p=7 ; echo \$((p+9))}
}

\vskip 10pt

\vbox{\head{Shell programming}
More shell programming constructs.  These are mostly useful in scripts, but
they can also be used on the command line.  (PS Don't use the C shell for shell
programming.)
\vskip 5pt
\cmd{Conditional code with test\par}%
{if [ {\it test} ] ; then {\it commands} ; fi}
\cmd{Tests:}{}
\cmd{Name exists and is a file}{-f {\it name}}
\cmd{File or directory is readable}{-r {\it file}}
\cmd{File or directory is writable}{-w {\it file}}
\cmd{Name exists and is a directory}{-d {\it name}}
\cmd{String is zero length}{-z {\it string}}
\cmd{String is non-zero length}{-n {\it string}}
\cmd{String comparison, case sensitive (Use {\tt [ "x\$FOO" == "xBAR" ]}
in case a string starts with '{\tt -}')}%
{{\it string1} == {\it string2}}
\cmd{}{{\it string1} != {\it string2}}
\cmd{Number comparison}{{\it number1} -eq {\it number2}}
\cmd{\ \ and also}{-ne -lt -gt -le -ge}
\cmd{Looping}{while {\it test} ; do {\it commands} ; done}
\cmd{Selecting alternatives\par}%
{case {\it word} in {\it words} ; {\it pattern}) {\it commands} ;; esac}
}

\vskip 10pt

\vbox{\head{Awk}
Like perl, only smaller and faster.
\vskip 5pt
\cmd{Print the {\it n}th word on every line}{awk '\{print \${\it n}\}'}
}

\vskip 10pt

\vbox{\head{Sed}
The stream editor.  Performs operations on its input stream, transforms to its
output stream.
\vskip 5pt
\cmd{Substitute first occurrence of {\it text1} with {\it text2} on each
line\par}{sed 's/{\it text1}/{\it text2}/'}
}

\end{multicols}

\vbox{\head{Examples}
Some examples of shell script fragments.  In these examples, '{\tt ;}' can be
replaced with newline in your own scripts.
\vskip 5pt
\cmd{Find all files newer than one day that contain a particular word}%
{find . -type f -mtime -1 -print0 | xargs -0 grep {\it word}}
\cmd{Print your {\tt \$PATH} variable readably}%
{echo \$PATH |awk -F : '\{for (i=1;i<=NF;i++)print \$i\}'}
\cmd{Print the environment variables for a particular process (specific to
Linux kernel with /proc filesystem)\par}%
{< /proc/{\it pid}/environ xargs -0 -n 1 echo}
\cmd{Output a single quote in a single-quoted string}{'abc'"'"'def'}
\cmd{Output a double quote in a double-quoted string}%
{"abc$\backslash$"def" {\rm or} "abc"'"'"def"}
}

\vspace{\fill}
\copyright 2002-2004 Donald J.\ Bindner,
\copyright 2004 Russell Steicke
-- licensed under the terms of the GNU
Free Documentation License version 1.1 or later.
\end{document}


