

%%%%% END of processed input.

%\end{multicols}
\break
\head{Credits and thanks}
\vskip 1.5ex
Special thanks to the late Peter S. Tillier (U.K.) for helping me with
the first release of this FAQ file, and to Daniel Jana, Yisu Dong, and
others for their suggestions and corrections.

\vskip 1.5ex
For additional syntax instructions, including the way to apply editing
commands from a disk file instead of the command line, consult:

\begin{itemize}
  \item
  "sed \& awk, 2nd Edition," by Dale Dougherty and Arnold Robbins
  (O'Reilly, 1997)

  \item
  "UNIX Text Processing," by Dale Dougherty and Tim O'Reilly (Hayden
  Books, 1987)

  \item
  "GAWK: Effective awk Programming," 3d edition, by Arnold D. Robbins
  (O'Reilly, 2003) or at http://www.gnu.org/software/gawk/manual/
\end{itemize}

\vskip 1.5ex
To fully exploit the power of awk, one must understand "regular
expressions." For detailed discussion of regular expressions, see
"Mastering Regular Expressions, 3d edition" by Jeffrey Friedl (O'Reilly,
2006).

\vskip 1.5ex
The info and manual ("man") pages on Unix systems may be helpful (try
"man awk", "man nawk", "man gawk", "man regexp", or the section on
regular expressions in "man ed").

\vskip 1.5ex
USE OF '\bs{}t' IN awk SCRIPTS: For clarity in documentation, I have used
'\bs{}t' to indicate a tab character (0x09) in the scripts.  All versions of
awk should recognize this abbreviation.


\vspace{\fill}
This info copied almost verbatim from
\verb|http://www.pement.org/awk/awk1line.txt|
\end{document}


