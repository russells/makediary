\documentclass{article}
% Copyright 2002-2004 Donald Bindner
% Copyright 2004-2008 Russell Steicke
% Permission is granted to copy, distribute and/or modify this
% document under the terms of the GNU Free Documentation License,
% Version 1.1 or any later version published by the Free Software
% Foundation.

\usepackage{multicol}
\usepackage[T1]{fontenc}

\setlength{\textheight}{11 in}
\setlength{\textwidth}{7.5 in}
\setlength{\hoffset}{-2 in}
\setlength{\voffset}{-1 in}
\setlength{\footskip}{12 pt}
\setlength{\oddsidemargin}{1.5 in}
\setlength{\evensidemargin}{1.5 in}
\setlength{\topmargin}{13 pt}
\setlength{\headheight}{12 pt}
\setlength{\headsep}{0 in}

\setlength{\parindent}{0 in}

\ifx \pdfpagewidth \undefined
\else
 \pdfpagewidth=8.5in   % page width of PDF output
 \pdfpageheight=11in   % page height of PDF output
\fi

\newcommand{\ctrl}{C-}

%% % Right align last line of paragraphs.
%% % This works, except that it right aligns every paragraph, including the
%% % headings.  And it also right aligns the only line of short paragraphs.
%% % Can it be put inside an environment?
%% % http://texnik.de/cgi-bin/mainFAQ.cgi?file=paragraph/paragraph
%% \leftskip=0pt plus 1fil
%% \rightskip=-\leftskip
%% \parfillskip=\leftskip
%% \parindent=0pt


% This is intended to be a list environment suitable for typesetting one
% item of the sed examples.  This compiles fine, but when I used it I get
% an error from LaTeX:
%
%  ! LaTeX Error: Something's wrong--perhaps a missing \item.
%
%  See the LaTeX manual or LaTeX Companion for explanation.
%  Type  H <return>  for immediate help.
%   ...
%
%  l.77 \begin{sedlist}
%                      \item[{\ensuremath\bullet}] double space a file
\newenvironment{sedlist}{
  \begin{list}
    {\ensuremath\bullet}
    { \setlength{\topsep}{0pt}
      \setlength{\partopsep}{0pt}
      \setlength{\parsep}{0pt}
      \slshape
    }
  \end{list}
}



\begin{document}
\pagestyle{empty}
\fontsize{9}{10}\selectfont

%% % Test the sedlist.
%% \begin{sedlist}
%% \item foo
%% \item bar
%% \end{sedlist}

\newcommand{\cmd}[2]{{\ensuremath\bullet~}#1:~\hfill\texttt{#2}\par\vskip2pt}
\newcommand{\head}[1]{{\large\textbf{#1}}\hfill\par
\vskip 2pt
\hrule
\vskip 4pt}
\newcommand{\bs}{\ensuremath\backslash}

%\begin{multicols}{2}

%\vskip 15pt

HANDY ONE-LINE SCRIPTS FOR AWK{\hfill}30 April 2008\\
Compiled by Eric Pement - eric [at] pement.org{\hfill}version 0.27

Latest version of this file (in English) is usually at:
   http://www.pement.org/awk/awk1line.txt

This file will also be available in other languages:
   Chinese  - http://ximix.org/translation/awk1line\_zh-CN.txt

\vskip 1.5ex
USAGE:
\vskip 1.5ex
\begin{tabular}{rll}
   Unix: & \texttt{awk '/pattern/ \{print "\$1"\}'}    & standard Unix shells \\
DOS/Win: & \texttt{awk '/pattern/ \{print "\$1"\}'}    & compiled with DJGPP, Cygwin \\
         & \texttt{awk "/pattern/ \{print \bs{}"\$1\bs{}"\}"}  & GnuWin32, UnxUtils, Mingw \\
\end{tabular}

\vskip 1.5ex
Note that the DJGPP compilation (for DOS or Windows-32) permits an awk
script to follow Unix quoting syntax \texttt{'/like/ \{"this"\}'}. HOWEVER, if the
command interpreter is CMD.EXE or COMMAND.COM, single quotes will not
protect the redirection arrows (<, >) nor do they protect pipes (|).
These are special symbols which require "double quotes" to protect them
from interpretation as operating system directives. If the command
interpreter is bash, ksh or another Unix shell, then single and double
quotes will follow the standard Unix usage.

\vskip 1.5ex
Users of MS-DOS or Microsoft Windows must remember that the percent
sign (\%) is used to indicate environment variables, so this symbol must
be doubled (\%\%) to yield a single percent sign visible to awk.

\vskip 1.5ex
If a script will not need to be quoted in Unix, DOS, or CMD, then I
normally omit the quote marks. If an example is peculiar to GNU awk,
the command 'gawk' will be used. Please notify me if you find errors or
new commands to add to this list (total length under 65 characters). I
usually try to put the shortest script first. To conserve space, I
normally use '1' instead of '\{print\}' to print each line. Either one
will work.

\vskip 1.5ex

%%%%% Start of processed input.
