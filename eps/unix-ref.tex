\documentclass{article}
% $Id: vi-back.tex,v 1.8 2002/10/09 18:53:32 dbindner Exp $
% Copyright 2002-2004 Donald Bindner
% Copyright      2004 Russell Steicke
% Permission is granted to copy, distribute and/or modify this
% document under the terms of the GNU Free Documentation License,
% Version 1.1 or any later version published by the Free Software
% Foundation.

\usepackage{multicol}

\setlength{\textheight}{11 in}
\setlength{\textwidth}{7.5 in}
\setlength{\hoffset}{-2 in}
\setlength{\voffset}{-1 in}
\setlength{\footskip}{12 pt}
\setlength{\oddsidemargin}{1.5 in}
\setlength{\evensidemargin}{1.5 in}
\setlength{\topmargin}{13 pt}
\setlength{\headheight}{12 pt}
\setlength{\headsep}{0 in}

\setlength{\parindent}{0 in}

\ifx \pdfpagewidth \undefined
\else
 \pdfpagewidth=11in    % page width of PDF output
 \pdfpageheight=8.5in  % page height of PDF output
\fi

\newcommand{\ctrl}{C-}

\begin{document}
\thispagestyle{empty}
\fontsize{9}{10}\selectfont

\newcommand{\cmd}[2]{#1 \hfill \texttt{#2}\par}
\newcommand{\head}[1]{{\large\textbf{#1}}\\}

\begin{multicols}{2}
%{\Large Unix commands}

%\vskip 15pt

\vbox{\head{Getting help}
Yes, there is help available.\par
\vskip 5pt
\cmd{Search the man pages for a keyword}{man -k {\it word}}
\cmd{Read the man page for a subject}{man {\it subject}}
\cmd{Read a man page in a particular section}{man {\it n} {\it subject}}
\cmd{(Man sections: 1=User commands 2=System calls 3=Functions 5=File formats
8=Admin commands)}{}
\cmd{Info system (mostly for programs from the GNU project)\par}%
{info {\it subject}}
}

\vskip 10pt

\vbox{\head{Directory navigation}
Moving around the file system.\par
\vskip 5pt
\cmd{Where am I?}{pwd}
\cmd{Change directory}{cd {\it dir}}
\cmd{Go to home directory}{cd}
\cmd{Up one level}{cd ..}
\cmd{Return to prev dir}{cd -}
}

\vskip 10pt

\vbox{\head{Directory listing}
Listing files in your file system.\par
\vskip 5pt
\cmd{Basic file listing}{ls}
\cmd{File listing including hidden (dot) files}{ls -a}
\cmd{Long listing (includes time, owner, etc)}{ls -l}
\cmd{List including char for file type}{ls -F}
}

\vskip 10pt

\vbox{\head{Finding files}
Finding files and operating on the files you do find.\par
\vskip 5pt
\cmd{Find files named {\it pattern}}{find {\it dir} -name {\it pattern}}
\cmd{Find files, run command on them\par}%
{find {\it dir} -name {\it pattern} | xargs {\it command}}
}

\vskip 10pt

\vbox{\head{Managing free space}
Commands to find out how much space is used or free.\par
\vskip 5pt
\cmd{List space used on mounted file systems}{df}
\cmd{List space used on one file system}{df \it filesystem}
\cmd{List space used in human-readable format}{df -h}
\cmd{List space used in a directory}{du \it dir}
\cmd{Summary of space used in a directory}{du -s \it dir}
\cmd{Summary in human-readable format}{du -hs \it dir}
}

\vskip 10pt

\vbox{\head{File system management}
Commands to mount, unmount and check file systems.\par
\vskip 5pt
\cmd{Mount a file system listed in fstab}{mount \it filesystem}
\cmd{}{mount \it device}
\cmd{Mount a file system}{mount -t {\it fstype} {\it device} {\it dir}}
\cmd{Unmount a mounted file system}{umount \it filesystem}
\cmd{Check an unmounted file system}{fsck -t {\it type} {\it device}}
\cmd{Create a file system on a device}{mkfs -t {\it type} {\it device}}
\cmd{List space used in human-readable format}{df -h}
\cmd{List space used in a directory}{du \it dir}
\cmd{Summary of space used in a directory}{du -s \it dir}
}

\vskip 10pt

\vbox{\head{Managing users}
Commands to manage users and passwords.\par
\vskip 5pt
\cmd{Change your own password}{passwd}
\cmd{Change password for a user (root only)}{passwd \it username}
\cmd{Change user id}{su \it username}
\cmd{Change user id, run login shell}{su - \it username}
\cmd{Change user id to root}{su}
\cmd{Run single command as root}{sudo \it command}
\cmd{Run single command, with shell redirection (your shell is running as you,
so it may not be able to open some files to do redirections)\par}%
{sudo sh -c '{\it command} > {\it f}'}
}

\vskip 10pt

\vbox{\head{Editors}
Text editors.\par
\vskip 5pt
\cmd{Visual editor, available everywhere}{vi [{\it filename}]}
\cmd{Vi IMproved, on most GNU/Linux systems}{vim}
\cmd{Emacs, a kitchen sink that is also a text editor}{emacs}
\cmd{Simpler text editor, from pine}{pico}
\cmd{Pico clone}{nano}
}

\vskip 10pt

\vbox{\head{Archiving}
Commands to archive files and directories.\par
\vskip 5pt
\cmd{Create a tar archive}{tar cf {\it archivefile.tar} {\it dirs}}
\cmd{Extract contents of a tar archive}{tar xf {\it archivefile.tar}}
\cmd{Create a compressed tar archive}{tar czf {\it archivefile.tar.gz} {\it dirs}}
\cmd{Copy a directory tree\par}%
{( cd {\it src} \&\& tar cf - . ) | ( cd {\it dest} \&\& tar xf -)}
}

\vskip 10pt

\vbox{\head{Mail}
Commands to send and read mail.  The Unix facilities for reading mail are
surprisingly rich, and often much more convenient than graphical mail
readers.\par
\vskip 5pt
\cmd{Send mail from command line}{mail -s {\it subject} {\it recipients}}
\cmd{}{{\it commands} | mail -s {\it subject} {\it recipients}}
\cmd{Popular text mode mail reader}{mutt}
\cmd{Send mail with mutt from command line\par}%
{echo hello | mutt -x -s {\it subject} [-a {\it file}] {\it recipients}}
\cmd{Older (non-free) text mode mail reader}{pine}
}

\vskip 10pt

\vbox{\head{Environment variables}
Setting and using environment variables.\par
\vskip 5pt
\cmd{Setting a variable.  {\it value} must be quoted if it contains white space
or shell metachars.}{VAR={\it value}}
\cmd{Getting a variable's value}{\$VAR {\rm or} "\$VAR"}
\cmd{Expand shell script arguments separately}{"\$@"}
\cmd{Substitute default value if {\tt VAR} is unset}{\$\{VAR:-{\it default}\}}
\cmd{Assign substitute value if {\tt VAR} is unset}{\$\{VAR:={\it default}\}}
\cmd{Assign shell variable array}{A=({\it element1} {\it element2} ...)}
\cmd{Subscript shell array}{\$\{A[{\it n}]\}}
\cmd{Expand entire array}{\$\{A[*]\}}
\cmd{Expand array, quoting each element separately}{"\$\{A[@]\}"}
}

\vskip 10pt

\vbox{\head{Bourne shell}
Hints for using the shell.  Here are some of the constructs that let you create
powerful shell programs, and they can all be used on the command line.\par
\vskip 5pt
\cmd{Run command in background}{{\it command} \&}
\cmd{Redirect output}{{\it command} > {\it file}}
\cmd{Redirect input}{{\it command} < {\it file}}
\cmd{Redirect output and error output}{{\it command} >{\it file} 2>\&1}
\cmd{Run command2 only if command1 succeeds (0 exit status)\par}%
{{\it command1} \&\& {\it command2}}
\cmd{Run command2 only if command1 fails (non-0 exit status)\par}%
{{\it command1} || {\it command2}}
\cmd{Force a command to succeed}{{\it command} || :}
\cmd{Run commands in subshell}{( {\it commands} )}
\cmd{Pipe between two subshells}{( {\it commands} ) | ( {\it commands} )}
\cmd{Substitute command output}{` {\it command} ` {\rm or} \$( {\it command} )}
\cmd{Shell arithmetic}{\$((3*4))}
\cmd{}{p=7 ; echo \$((p+9))}
}

\vskip 10pt

\vbox{\head{Bourne shell programming}
More shell programming constructs.  These are mostly useful in scripts, but
they can also be used on the command line.  (PS Don't use the C shell for shell
programming.)
\vskip 5pt
\cmd{Conditional code with test\par}%
{if [ {\it test} ] ; then {\it commands} ; fi}
\cmd{Tests:}{}
\cmd{Name exists and is a file}{-f {\it name}}
\cmd{File or directory is readable}{-r {\it file}}
\cmd{File or directory is writable}{-w {\it file}}
\cmd{Name exists and is a directory}{-d {\it name}}
\cmd{String is zero length}{-z {\it string}}
\cmd{String is non-zero length}{-n {\it string}}
\cmd{String comparison, case sensitive (Use {\tt [ "x\$FOO" == "xBAR" ]}
in case a string starts with '{\tt -}')}%
{{\it string1} == {\it string2}}
\cmd{}{{\it string1} != {\it string2}}
\cmd{Number comparison}{{\it number1} -eq {\it number2}}
\cmd{\ \ and also}{-ne -lt -gt -le -ge}
\cmd{Looping}{while {\it test} ; do {\it commands} ; done}
\cmd{Selecting alternatives\par}%
{case {\it word} in {\it words} ; {\it pattern}) {\it commands} ;; esac}
}

\end{multicols}

\vspace{\fill}
\copyright 2002-2004 Donald J.\ Bindner,
\copyright 2004 Russell Steicke
-- licensed under the terms of the GNU
Free Documentation License version 1.1 or later.
\end{document}


