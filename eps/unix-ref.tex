\documentclass{article}
% $Id: vi-back.tex,v 1.8 2002/10/09 18:53:32 dbindner Exp $
% Copyright 2002-2004 Donald Bindner
% Copyright      2004 Russell Steicke
% Permission is granted to copy, distribute and/or modify this
% document under the terms of the GNU Free Documentation License,
% Version 1.1 or any later version published by the Free Software
% Foundation.

\usepackage{multicol}

\setlength{\textheight}{11 in}
\setlength{\textwidth}{7.5 in}
\setlength{\hoffset}{-2 in}
\setlength{\voffset}{-1 in}
\setlength{\footskip}{12 pt}
\setlength{\oddsidemargin}{1.5 in}
\setlength{\evensidemargin}{1.5 in}
\setlength{\topmargin}{13 pt}
\setlength{\headheight}{12 pt}
\setlength{\headsep}{0 in}

\setlength{\parindent}{0 in}

\ifx \pdfpagewidth \undefined
\else
 \pdfpagewidth=11in    % page width of PDF output
 \pdfpageheight=8.5in  % page height of PDF output
\fi

\newcommand{\ctrl}{C-}

\begin{document}
\thispagestyle{empty}
\fontsize{9}{10}\selectfont

\newcommand{\key}[2]{#1 \hfill \texttt{#2}\par}
\newcommand{\head}[1]{{\large\textbf{#1}}\\}

\begin{multicols}{2}
{\Large Unix commands}

\vskip 15pt

\vbox{\head{Directory navigation}
Moving around the file system.\par
\vskip 5pt
\key{where am I?}{pwd}
\key{change directory}{cd}
\key{up one level}{cd ..}
\key{return to prev dir}{cd -}
}

\vskip 10pt

\vbox{\head{Directory listing}
Listing files in your file system.\par
\vskip 5pt
\key{Basic file listing}{ls}
\key{File listing including hidden (dot)files}{ls -a}
\key{Long listing (includes time, owner, etc)}{ls -l}
\key{List including char for file type}{ls -F}
}

\vskip 10pt

\vbox{\head{Managing free space}
Commands to find out how much space is used or free.\par
\vskip 5pt
\key{List space used on mounted file systems}{df}
\key{List space used on one file system}{df \it filesystem}
\key{List space used in human-readable format}{df -h}
\key{List space used in a directory}{du \it dir}
\key{Summary of space used in a directory}{du -s \it dir}
}

\vskip 10pt

\vbox{\head{File system management}
Commands to mount, unmount and check file systems.\par
\vskip 5pt
\key{Mount a file system listed in fstab}{mount \it filesystem}
\key{}{mount \it device}
\key{Mount a file system}{mount -t {\it fstype} {\it device} {\it dir}}
\key{Unmount a mounted file system}{umount \it filesystem}
\key{List space used in human-readable format}{df -h}
\key{List space used in a directory}{du \it dir}
\key{Summary of space used in a directory}{du -s \it dir}
}

\vskip 10pt

\vbox{\head{Managing users}
Commands to manage users and passwords.\par
\vskip 5pt
\key{Change your own password}{passwd}
\key{Change password for a user (root only)}{passwd \it username}
\key{Change user id}{su \it username}
\key{Change user id, run login shell}{su - \it username}
\key{Change user id to root}{su}
\key{Run single command as root}{sudo \it command}
\key{Run single command, with shell redirection\par}{sudo sh -c '{\it command} > {\it f}'}
}

\vskip 10pt

\vbox{\head{Editors}
Text editors.\par
\vskip 5pt
\key{Visual editor, available everywhere}{vi}
\key{Vi IMproved, on most GNU/Linux systems}{vim}
\key{Emacs, a kitchen sink that is also a text editor}{emacs}
\key{Simpler text editor, from pine}{pico}
\key{Pico clone}{nano}
}

\vskip 10pt

\vbox{\head{Archiving}
Commands to archive files and directories.\par
\vskip 5pt
\key{Create a tar archive}{tar cf {\it archivefile.tar} {\it dirs}}
\key{Create a compressed tar archive}{tar czf {\it archivefile.tar.gz} {\it dirs}}
}

\vskip 10pt

\vbox{\head{Mail}
Commands to send and read mail.  The Unix facilities for reading mail are
surprisingly rich, and often much more convenient than graphical mail
readers.\par
\vskip 5pt
\key{Send mail from command line}{mail -s {\it subject} {\it recipients}}
\key{}{{\it commands} | mail -s {\it subject} {\it recipients}}
\key{Popular text mode mail reader}{mutt}
\key{Send mail with mutt from cmd line\par}%
{mutt -x -s {\it subject} [-a {\it file}] {\it recipients}}
\key{Older (non-free) text mode mail reader}{pine}
}

\vskip 10pt

\vbox{\head{Shell tricks}
Hints for using the shell.  Here are some of the constructs that let you create
powerful shell programs, and they can all be used on the command line.\par
\vskip 5pt
\key{Redirect output}{{\it command} > {\it file}}
\key{Redirect input}{{\it command} < {\it file}}
\key{Redirect output and error output}{{\it command} >{\it file} 2>\&1}
\key{Run command2 only if command1 succeeds (0 exit status)\par}%
{{\it command1} \&\& {\it command2}}
\key{Run command2 only if command1 fails (non-0 exit status)\par}%
{{\it command1} || {\it command2}}
\key{Force a command to succeed}{{\it command} || :}
\key{Run commands in subshell}{( {\it commands} )}
\key{Pipe between two subshells}{( {\it commands} ) | ( {\it commands} )}
}

\vskip 10pt

\vbox{\head{Shell programming}
More shell programming constructs.  These are mostly useful in scripts, but
they can also be used on the command line.
\vskip 5pt
\key{Conditional code with test\par}{if [ {\it test} ] ; then {\it commands} ; done}
\key{Tests:}{}
\key{Name exists and is a file}{-f {\it name}}
\key{File or directory is readable}{-r {\it file}}
\key{File or directory is writable}{-w {\it file}}
\key{Name exists and is a directory}{-d {\it name}}
\key{String is zero length}{-z {\it string}}
\key{String is non-zero length}{-n {\it string}}
}

\end{multicols}

\vspace{\fill}
\copyright 2002-2004 Donald J.\ Bindner,
\copyright 2004 Russell Steicke
-- licensed under the terms of the GNU
Free Documentation License version 1.1 or later.
\end{document}


% -----------------------

\vskip 10pt

\vbox{\head{Objects}
In addition to the usual motions that are supported by Vi, Vim supports
several objects: word, sentence, paragraph, and block.  These are used with
editing commands (delete, change, and yank).  The advantage over
motions is that you do not have to be at one end of an object to use it.
For example \texttt{das} will delete a sentence from anywhere within it.\par
\vskip 5pt
\key{a word, a sentence, a paragraph}{aw, as, ap}
\key{same without white space (inner)}{iw, is, ip}
\key{blocks delimited by $[\,]$, (), \texttt{<>}, and $\{\}$}{a[, a(, a<, a$\{$}
\key{same without surrounding brackets}{i[, i(, i<, i$\{$}
}

\vskip 10pt

\vbox{\head{Extra motions}
\key{jump to beginning, end of $\{\,\}$ block}{[$\{$ , ]$\}$}
\key{same for parentheses}{[( , ])}
\key{same for / (C--style comments)}{[/ , ]/}
\key{previous end of word, space--delimited}{ge, gE}
}

\vskip 10pt

\vbox{\head{Extra markers}
Marks set with capital letters are ``global''; jumping to a global mark may
take you to a different file.  Several marks are remembered between
editing sessions:
\vskip 5pt
\key{position when file was last open}{'"}
\key{position where file was last edited}{'.}
\key{file and position of last Vim session}{'0}
}

\vskip 10pt

\vbox{\head{Other searching}
\key{incremental search}{:set (no)incsearch}
\key{highlight search}{:set (no)hlsearch}
\key{ignore case in search}{:set (no)ignorecase}
\key{find word under cursor fwd, back}{* , \#}
}

\vskip 10pt

\vbox{\head{Interface and colors}
\key{syntax coloring}{:syntax on}
\key{switch to graphical}{:gui}
\vskip 5pt
\key{show line numbers}{:set number}
\key{show commands in progress}{:set showcmd}
\key{show matching bracket}{:set showmatch}
\vskip 5pt
\key{autoindent}{:set ai}
\key{smartindent}{:set si}
\key{file format (dos, unix, mac)}{:set ff=...}
\key{good for editing tables}{:set virtualedit=all}
\key{edit option set}{:options}
}

\vskip 10pt

\vbox{\head{Insert mode editing}
There are some commands available while in insert mode:\par
\vskip 5pt
\key{copy from line above, below}{\ctrl Y , \ctrl E}
\key{complete from previous match}{\ctrl P}
\key{insert from (paste) register}{\ctrl R}
\key{insert digraph}{\ctrl K \textit{di}}
\key{execute single command}{\ctrl O \textit{command}}
}

\vskip 10pt

\vbox{\head{Folding}
To make working with a file more convenient, multiple lines may be
``folded'' together into a single highlighted line.  Use this to get long
sections out of the way while editing.\par
\vskip 5pt
\key{create, delete a fold}{zf, zd}
\key{open, (re--)close a fold}{zo, zc}
\key{remove outermost folds, more folds}{zr, zm}
\key{remove all, re--fold all folds}{zR, zM}
\key{disable, (re--)enable, toggle folds}{zn, zN, zi}
\key{save, load view (including folds)}{:mkview, :lo}
}

\vskip 10pt

\vbox{\head{Compiling and source code}
There are several commands for facilitating software development:
\vskip 5pt
\key{correct indentation of line}{==}
\key{increase the indent level}{>>}
\key{decrease the indent level}{<<}
\vskip 5pt
\key{run make and move cursor to first error}{:make}
\key{move to next, previous error}{:cnext, :cprev}
\key{move to first, last error}{:cfirst, :clast}
\key{view error or list of errors}{:cc, :clist}
\key{list lines with identifier under cursor}{[I}
}

\vskip 10pt

\vbox{\head{Using ctags}
You must first run ctags or etags on your file to use these features.\par
\vskip 5pt
\key{jump to function with tagname}{:tag \textit{tagname}}
\key{jump to tag under cursor}{\ctrl ]}
\key{previous tag}{\ctrl t}
}

\vskip 10pt

\vbox{\head{Other}
\key{reformat margins}{gq}
\key{make case upper, lower, toggle}{gU, gu, g\~\ }
\key{add subtract from number under cursor}{\ctrl A , \ctrl X}
\key{edit file under cursor}{gf}
\key{view man page under cursor}{K}
\key{print highlighted copy}{:hardcopy}
\key{list digraphs}{:digraphs}
\key{editable command history}{q:}
}

\end{multicols}

\vspace{\fill}
\copyright 2002-2004 Donald J.\ Bindner,
\copyright 2004 Russell Steicke
-- licensed under the terms of the GNU
Free Documentation License version 1.1 or later.
\end{document}
